\chapter{Campos Potenciales}
\section{Campos Potenciales Atractivos}
\section{Campos Potenciales Repulsivos}
\section{Campos Potenciales Usando Sensores de Proximidad}
\subsection{Trayectorias}
\begin{definicion}

	Un \textit{árbol} es un conjunto parcialmente ordenado, $\left\langle T,\, \leq_{_T} \right\rangle$, tal que $\forall \, x\in T$, el conjunto $x_{\scriptscriptstyle <} = \left\lbrace y\in T \mid y <_{_T} x \right\rbrace$ es bien ordenado por $\leq_{_T}$. \\
	
	A los elementos de $T$ se les llaman nodos; y si $T$ tiene elemento mínimo, a este se le llama raíz del árbol.  
\end{definicion}

Dado un árbol $T$, para cada $x\in T$, $\, x_{\scriptscriptstyle <}$ es isomorfo a un único ordinal, $\textbf{\textit{alt}}(x)$, llamado la altura de $x$. El $\alpha$-ésimo nivel de $T$ es el conjunto: $\textit{\textbf{Niv}}_{\alpha}(T) = \left\lbrace x\in T \mid \textit{\textbf{alt}}(x) = \alpha \right\rbrace$, es decir, el conjunto de los elementos de $T$ que tienen altura $\alpha$.

\begin{definicion}
	Sea $T$ un \textit{árbol}. 
	\begin{itemize}
		\item[1.] Definimos la altura de $T$ como el mínimo ordinal $\alpha$ tal que $\textit{\textbf{Niv}}_{\alpha}(T) = \emptyset$. La altura de $T$ la denotamos por \textit{\textbf{alt}}($T$).
		
		\item[2.] Una rama en $T$ es una cadena maximal en $T$; y su longitud es su tipo de orden. Una rama de longitud igual a la altura del árbol es llamada rama cofinal.	
	\end{itemize}
\end{definicion}

\begin{definicion}
	Sea $T$ un \arbol. Un \textit{subárbol} de $T$ es un subconjunto $T'\subseteq T$ junto con el orden restringido a $T'$ tal que $\forall \, x\in T'\: \, \forall \, y\in T \left( y < x \ent y\in T'\right) $. 
\end{definicion} 


\begin{ejemplo}
	Si $\left\langle A, \leq \right\rangle \in BO$ entonces $A$ es un árbol. Note que \textit{\textbf{alt}}($A$) es el tipo de orden de $A$ y la única rama en $A$ es $A$ mismo, la cual es cofinal.  
\end{ejemplo} 

\begin{ejemplo}
	Sea $\lambda$ un ordinal y $A \neq \emptyset$. Definimos $A^{<\lambda} = \displaystyle \bigcup_{\alpha < \lambda} {^\alpha}A$. Así, $A^{<\lambda}$ es el conjunto de sucesiones transfinitas de elementos de $A$ de longitud menor que $\lambda$. Resulta que $\left\langle A^{<\lambda},\, \scriptstyle \subseteq \right\rangle$ es un \arbol. El caso $A = 2$, es conocido como el árbol binario completo de altura $\lambda$
\end{ejemplo}

\begin{teorema}[(\textit{\textbf{Lema de König}})][lemakonig]
	Si $T$ es un árbol tal que
	\begin{itemize}
		\item[1.] \textit{\textbf{alt}}($T$) = $\omega$ y
		\item[2.] $\forall \, \alpha, \, |\textit{\textbf{Niv}}_{\alpha}(T)| < \aleph_{\scriptscriptstyle 0}$,
	\end{itemize}
	entonces $T$ tiene una rama cofinal. 	
\end{teorema}

\textbf{Demostración:}\\

Como cada nivel en $T$ es finito y su altura es $\omega$, entonces $|T| = \aleph_{\scriptscriptstyle 0}$. Elegimos $x_{0} \in \textit{\textbf{Niv}}_{0}(T)$ tal que el conjunto $\{ \,y\in T\mid x_{0} \leq y \}$ es infinito; $x_{1} \in \textit{\textbf{Niv}}_{1}(T)$ de manera que $|\left\lbrace z\in T \mid x_{1} \leq z \right\rbrace | = \aleph_{\scriptscriptstyle 0}$. Este proceso lo podemos continuar ya que $T$ es infinito. Así, por recursión, tenemos el conjunto $\{x_{n} \mid n < \omega\}$ y cualquier cadena que extienda a esta es una rama cofinal. \\

\begin{definicion}
	Sea $\kappa$ un cardinal. Un $\kappa$-\textit{árbol} es un árbol de altura $\kappa$ cuyos niveles tienen cardinalidad menor que $\kappa$.
\end{definicion}

\begin{definicion}
	Un $\kappa$-\textit{árbol} de \textit{Aronszajn} es un $\kappa$-árbol que no tiene ramas cofinales. 
\end{definicion}

\begin{nota}
	Del lema de König tenemos que no hay $\aleph_{\scriptscriptstyle 0}$-árboles de Aronszajn. Sin embargo sí hay $\aleph_{\scriptscriptstyle 1}$-árboles de Aronszajn.
\end{nota} 

\begin{definicion}
	$\kappa$ tiene la propiedad arborescente si todo $\kappa$-árbol tiene una rama cofinal. 
\end{definicion} 

\begin{nota}
	De las definiciones se sigue que: $\kappa$ tiene la propiedad arborescente \sii no hay $\kappa$-árboles de Aronszajn.
\end{nota}

\begin{definicion}
	Sea $A$ un conjunto. Una partición de $A$ es una familia $\left\lbrace X_{i} \right\rbrace_{i\in I}$ de subconjuntos de $A$ tal que:
	\begin{itemize}
		\item[1.] $\forall \, i\in I:\, X_{i} \cap X_{j} = \emptyset$ si $i\neq j$.
		\item[2.] $A = \displaystyle \bigcup_{i\in I} X_{i}$.
	\end{itemize}
\end{definicion}

Note que dada una partición $\left\lbrace X_{i} \right\rbrace_{i\in I}$ de $A$ podemos asignarle la función $F: A \longrightarrow I$ dada por $F(a) = i$ \sii $a\in X_{i}$. Y si $F:A \longrightarrow I$ es una función podemos asignarle la partición $\left\lbrace F^{-1}(\left\lbrace i \right\rbrace ) \right\rbrace_{i\in I}$

\begin{definicion}
	Sean $A$ un conjunto y $n$ un cardinal. Definimos
	$$\left[ A\right]^{n} = \left\lbrace x\subseteq A \mid \left| x \right|  = n \right\rbrace $$
\end{definicion}

\begin{definicion}
	Sea $\mathscr{F} = \left\lbrace X_{i}\right\rbrace _{i\in I}$ una partición de $\left[ A\right]^{n}$. Un subcojunto $H \subseteq A$ es homogéneo para la partición $\mathscr{F}$ \sii existe $i\in I$ tal que $\left[  H \right]^{n} \subseteq X_{i}$; o bien \sii $\left|  F\left( \left[  H \right]^{n} \right)  \right| = 1$, donde $F$ es la función inducida por la partición.
\end{definicion}




------------begin----------




\begin{definicion}[(Compacidad débil)]
$\mathscr{L}_{\kappa \, \kappa}$ satisface compacidad débil si y solo si todo conjunto de enunciados $\Sigma \subseteq \mathscr{L}_{\kappa \, \kappa}$, $|\Sigma| = \kappa$, para el que todo subconjunto $\Gamma \subseteq \Sigma$, con $|\Gamma| < \kappa$, tiene modelo, entonces $\Sigma$ tiene modelo.  
\end{definicion}	 

\begin{definicion}
$\kappa$ es débilmente compacto si y solo si
\begin{itemize}
	\item[a)] $\kappa > \omega$,
	\item[b)] $\kappa$ es fuerte y
	\item[c)] $\mathscr{L}_{\kappa \, \kappa}$ satisface compacidad débil. 
\end{itemize}
\end{definicion}

\begin{teorema}
 Sea $\kappa > \omega$. Son equivalentes:
 \begin{itemize}
 	\item[1.] $\kappa$ es débilmente compacto.
 	\item[2.] $\mathscr{L}_{\kappa \, \omega}$ satisface compacidad débil. 
 	\item[3.] $\kappa \longrightarrow \displaystyle \left(\kappa\right)_{2}^{2}$.
 	\item[4.] $\kappa$ es inaccesible y satisface la propiedad arborescente.
 	\item[5.] No existen $\kappa$-árboles de Aronszajn.
 	\item[6.] $\kappa \longrightarrow \displaystyle \left(\kappa\right)_{m}^{n}, \, \forall m \in \omega \, \, \forall m < \kappa$.	
 \end{itemize}	
\end{teorema}

\begin{teorema}[(Silver)] 
	Si $\kappa$ es débilmente compacto, entonces $(\kappa \mbox{ es débilmente compacto})^L$.
\end{teorema}